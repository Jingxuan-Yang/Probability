\documentclass{elegantbook}

%%=====编译方式pdflatex============================
%%=====否则公式中数学字体不为times====================

%%=====新数学命令================================
\newcommand\dif{\mathrm{d}}
\newcommand\no{\noindent}
\newcommand\dis{\displaystyle}
\newcommand\ls{\leqslant}
\newcommand\gs{\geqslant}

\newcommand\limit{\dis\lim\limits}
\newcommand\limn{\dis\lim\limits_{n\to\infty}}
\newcommand\limxz{\dis\lim\limits_{x\to0}}
\newcommand\limxi{\dis\lim\limits_{x\to\infty}}
\newcommand\limxpi{\dis\lim\limits_{x\to+\infty}}
\newcommand\limxni{\dis\lim\limits_{x\to-\infty}}

\newcommand\sumn{\dis\sum\limits_{n=1}^{\infty}}
\newcommand\sumnz{\dis\sum\limits_{n=0}^{\infty}}

\newcommand\sumi{\dis\sum\limits_{i=1}^{\infty}}
\newcommand\sumin{\dis\sum\limits_{i=1}^{n}}
\newcommand\sumizn{\dis\sum\limits_{i=0}^{n}}

\newcommand\sumk{\dis\sum\limits_{k=1}^{\infty}}
\newcommand\sumkz{\dis\sum\limits_{k=0}^{\infty}}
\newcommand\sumkn{\dis\sum\limits_{k=0}^n}
\newcommand\sumkfn{\dis\sum\limits_{k=1}^n}

\newcommand\pzx{\dis\frac{\partial z}{\partial x}}
\newcommand\pzy{\dis\frac{\partial z}{\partial y}}

\newcommand\pfx{\dis\frac{\partial f}{\partial x}}
\newcommand\pfy{\dis\frac{\partial f}{\partial x}}

\newcommand\pzxx{\dis\frac{\partial^2 z}{\partial x^2}}
\newcommand\pzxy{\dis\frac{\partial^2 z}{\partial x\partial y}}
\newcommand\pzyx{\dis\frac{\partial^2 z}{\partial y\partial x}}
\newcommand\pzyy{\dis\frac{\partial^2 z}{\partial y^2}}

\newcommand\pfxx{\dis\frac{\partial^2 f}{\partial x^2}}
\newcommand\pfxy{\dis\frac{\partial^2 f}{\partial x\partial y}}
\newcommand\pfyx{\dis\frac{\partial^2 f}{\partial y\partial x}}
\newcommand\pfyy{\dis\frac{\partial^2 f}{\partial y^2}}

\newcommand\intzi{\dis\int_{0}^{+\infty}}
\newcommand\intd{\dis\int}
\newcommand\intab{\dis\int_a^b}
\newcommand\ma{\mathcal{A}}
\newcommand\mr{\mathcal{R}}
%%==============================================

%%=====定义新数学符号=====================
\DeclareMathOperator{\sgn}{sgn}
\DeclareMathOperator{\arccot}{arccot}
\DeclareMathOperator{\arccosh}{arccosh}
\DeclareMathOperator{\arcsinh}{arcsinh}
\DeclareMathOperator{\arctanh}{arctanh}
\DeclareMathOperator{\arccoth}{arccoth}
\DeclareMathOperator{\grad}{\bf{grad}}
%====================================

\author{杨敬轩}
\date{\today}
\email{JingXuanHuan.Yang@gmail.com}
\usepackage{ntheorem}
\zhtitle{概率论}
%\zhend{模板}
\entitle{Probability}
%\enend{Template}
\version{\LaTeX\ v1.0}
\myquote{Victory won\rq t come to us unless we go to it.}
\logo{logop.jpg}
\cover{cover.pdf}

%green color
   \definecolor{main1}{RGB}{0,120,2}
   \definecolor{second1}{RGB}{230,90,7}
   \definecolor{third1}{RGB}{0,160,152}
%cyan color
   \definecolor{main2}{RGB}{0,175,152}
   \definecolor{second2}{RGB}{239,126,30}
   \definecolor{third2}{RGB}{120,8,13}
%blue color
   \definecolor{main3}{RGB}{20,50,104}
   \definecolor{second3}{RGB}{180,50,131}
   \definecolor{third3}{RGB}{7,127,128}

\usepackage{makecell}
\usepackage{lipsum}
\usepackage{texnames}

\begin{document}
\maketitle
\tableofcontents
\mainmatter
\chapter{Axioms of Probability}

\begin{definition}{Sample Space}{Sample}
The sample space $\Omega$ of an experiment is the set of all possible outcomes of the experiment.
\end{definition}

\begin{definition}{Event}{Event}
An event of an experiment is a subset of the sample space $\Omega$ of the experiment. 
We call $\Omega$ the certain event and $\Phi$ the impossible event of the experiment. 
We say that an event $A$ occurs if the outcome of the experiment belongs to $A$.

\end{definition}

\begin{definition}{$\sigma$-algebra}{algebra}
A $\sigma$-algebra $\ma$ of subsets of a sample space $\Omega$ is a collection of subset of $\Omega$ s.t.\\
(1) $\Omega\in\ma$,\\
(2) $\ma$  is closed under complementation, i.e., if $A\in\ma$ , then $\Omega\backslash A\in\ma$,\\
(3) $\ma$  is closed under countable union, i.e., if $A_n\in\ma$ for $n=1,2,\cdots$, then $$\bigcap_{n=1}^\infty A_n \in\ma.$$

\end{definition}

\begin{theorem}{Properties of $\sigma$-algebra}{Properties}
Suppose $A$ is a $\sigma$-algebra of subsets of a sample space $\Omega$.
\\
(1) $\Phi\in\ma,$\\
(2) $A$ is closed under finite union,\\
(3) $A$ is closed under countable and finite intersection.

\end{theorem}

\begin{theorem}{Intersection of $\sigma$-algebra}{Intersection}
Suppose $\Gamma$ is a nonempty collection of $\sigma$-algebra of subsets of a sample space $\Omega$. Then the intersection $$B=\bigcap_{A\in\Gamma}A$$ of the $\sigma$-algebra in $\Gamma$ is also a $\sigma$-algebra of subsets of $\Omega$.

\end{theorem}

\begin{corollary}{Existence of Smallest $\sigma$-algebra}{Existence}
Suppose $\mathcal{C}$ is a collection of subsets of a sample space $\Omega$. Then there exists a smallest $\sigma$-algebra of subsets of $\Omega$ including $\mathcal{C}$.

\end{corollary}

\begin{definition}{Generated $\sigma$-algebra}{Generated}
Let $\mathcal{C}$ be a collection of subsets of a sample space $\Omega$, we define the $\sigma$-algebra of subsets of $\Omega$ generated by $\mathcal{C}$ as the smallest $\sigma$-algebra of subsets of $\Omega$ including $\mathcal{C}$ and denoted it as $\sigma$($\mathcal{C}$).

\end{definition}

\begin{definition}{Probability Measure}{Probability Measure}
Let $A$ be a $\sigma$-algebra of subsets of a sample space $\Omega$, a probability measure $P: \ma\rightarrow\mr$ on $A$ is a real-valued function on $A$ s.t.\\
(1) Nonnegativity: $P(A)\gs0,\ \forall A\in\ma,$ \\
(2) Normalization: $P(\Omega)=1$, \\
(3) Countable additivity: If $A_1,A_2, \cdots$ are pairwise disjoint events in 
$A$ then $$P\left(\bigcup_{n=1}^\infty A_n \right)=\sumn P(A) .$$
For an event $A\in\ma$ , we call $P(A)$ the probability of the event $A$.

\end{definition}

\begin{definition}{Probability Space}{Probability Space}
A probability space is an ordered triple $(\Omega,\ma, P)$ consisting of a sample space $\Omega$ , a $\sigma$-algebra $\ma$  of subsets of $\Omega$, and a probability measure $P$ on $\ma$ .

\end{definition}

\begin{theorem}{A Kind of Probability Measure }{A Kind}
Suppose $\Omega={w_1,w_2,\cdots}$, $\ma\in\mathcal{P}(\Omega)$ and $$P(A)=\sum\limits_{w_i\in\ma}P_i,\  \text{for all}\ A\in\mathcal{P}(\Omega),$$
where $P_i\gs0,\ \forall i=1,2,\cdots$ and $$\sumi P_i =1,$$ then $P$ is a probability measure on $\mathcal{P}(\Omega)$. A similar result holds if $\Omega={w_1,w_2,\cdots,w_N}$, where $N\gs1$.

\end{theorem}

\begin{corollary}{A Kind of Probability Measure (special)}{special}
Suppose $\Omega={w_1,w_2,\cdots,w_N}$, $\ma\in\mathcal{P}(\Omega)$, and $$P(A)=\frac{|A|}{N}$$ for all $A\in\mathcal{P}(\Omega)$, then $P$ is a probability measure on $\mathcal{P}(\Omega)$.

\end{corollary}

\begin{theorem}{Classical definition of probability}{Classical}
Suppose $\Omega={w_1,w_2,\cdots,w_N}$, $\ma\in\mathcal{P}(\Omega)$ and $P$ is a probability measure on $\mathcal{P}(\Omega)$  such that $P({w_1})= P({w_2})=\cdots= P({w_N})$, then $$P(A)=\frac{|A|}{N}$$ for all $A\in\mathcal{P}(\Omega)$.

\end{theorem}

\begin{theorem}{Properties of Probability Measure}{Properties}
Suppose $(\Omega,\ma, P)$ is a probability space.\\
(1) $P(\Phi)=0,$\\
(2) $P(A)+P(A^c)=1$. Therefore, $0\ls P(A)\ls1$, for all $A\in\ma$.\\
(3) Finite additivity: If $A_1,A_2,\cdots,A_N$ are pairwise disjoint events in $A$,              then $$P\left(\bigcup_{n=1}^NA_n \right)=\sum_{n=1}^NP(A). $$

\end{theorem}

\begin{theorem}{Properties of Probability Measure}{Properties of P}
Suppose $(\Omega,\ma, P)$ is a probability space, and suppose $A,B\in\ma$.
\\
(1) If $A_1,A_2,\cdots$ are pairwise disjoint events on $A$ and 
$$ \bigcup_{n=1}^\infty A_n =\Omega,$$ then 
$$P(A)= \sumi P(A \cap A_n ).$$
(2) If $B\subseteq A$, then $P(A)=P(A\cap B)+P(A\cap A^c )$ for all $A,B\in\ma$.\\
(3) $P(A\cap B)  \ls \min \{P(A),P(B)\}  \ls \max\{ P(A),P(B)\} \ls  P(A\cup B)$.

\end{theorem}

\begin{corollary}{Finite Additivity under Union}{Finite Additivity}
Suppose $(\Omega,\ma, P)$ is a probability space, $A\in\ma$ , $A_1,A_2,\cdots$ are pairwise disjoint events in $\ma$, and $$P\left(\bigcup_{n=1}^\infty A_n \right)=1,$$ then 
$$P(A)=\sum_{n=1}^\infty P\left(A\cap A_n \right).$$ 

\end{corollary}

\begin{theorem}{Inclusion-exclusion identity}{Inclusion-exclusion identity}
Suppose $(\Omega,\ma, P)$ is a probability space, and suppose $A_1,A_2,\cdots,A_n\in\ma $, where $n \gs2$, then $$P\left(\bigcup_{i=1}^n A_i \right)=\sumkfn(-1)^{k+1}\cdot  \sum_{1\ls i_1<i_2<\cdots<i_k\ls n}P\left(A_{i_1} \bigcap A_{i_2 } \bigcap\cdots\bigcap A_{i_k } \right) .$$

\end{theorem}

\begin{lemma}{Generated Pairwise Disjoint}{Generated Pairwise Disjoint}
Suppose $A$ is a $\sigma$-algebra of subsets of a sample space $\Omega$, suppose $A_1,A_2,\cdots\in\ma$, $B_1=A_1$, and $$B_n=A_n\setminus\bigcup_{i=1}^{n-1}A_i$$  for all $n\gs2$, then $B_1,B_2,\cdots$ are pairwise disjoint events in $\ma$, $$\bigcup_{i=1}^nA_i =\bigcup_{i=1}^nB_i$$  for all $n\gs1$, and $$\bigcup_{n=1}^\infty A_n =\bigcup_{n=1}^\infty B_n.$$

\end{lemma}

\begin{theorem}{Inclusion-exclusion inequality}{Inclusion-exclusion inequality}
Suppose $(\Omega,\ma, P)$ is a probability space, and suppose $A_1,A_2,\cdots,A_n\in\ma$, where $n\gs2$, then
$$P\left(\bigcup_{i=1}^n A_i \right)
\begin{cases}
\ls\sum\limits_{k=1}^m(-1)^{k+1}\cdot  \sum\limits_{1\ls i_1<i_2<\cdots<i_k\ls n}P\left(A_{i_1} \bigcap A_{i_2 } \bigcap\cdots\bigcap A_{i_k } \right),& \text{if $m$ is odd,}\\
\gs\sum\limits_{k=1}^m(-1)^{k+1}\cdot  \sum\limits_{1\ls i_1<i_2<\cdots<i_k\ls n}P\left(A_{i_1} \bigcap A_{i_2 } \bigcap\cdots\bigcap A_{i_k } \right),& \text{if $m$ is even,}
\end{cases}
$$
where $1\ls m\ls n$.\\
In particular,
$$P\left(\bigcup_{i=1}^n A_i \right)\ls\sumin P(A_i),$$
$$P\left(\bigcup_{i=1}^n A_i \right)\gs\sumin P(A_i)-
\sum\limits_{1\ls i<j\ls n}P\left(A_i\bigcap A_j\right).$$
\end{theorem}

\begin{theorem}{Boole's inequality}{Boole's inequality}
Suppose $(\Omega,\ma, P)$ is a probability space, and suppose $A_1,A_2,\cdots\in\ma$ , then 
$$P\left(\bigcup_{i=1}^\infty A_i \right)\ls\sum_{i=1}^\infty P(A_i ) .$$

\end{theorem}

\begin{definition}{Monotonicity}{Monotonicity}
Let $(\Omega,\ma, P)$ be a probability space.\\
A sequence $\{A_1,A_2,\cdots\}$ of events in $A$ is increasing if $A_1\subseteq A_2\subseteq\cdots$\\
A sequence $\{A_1,A_2,\cdots\}$ of events in $A$ is decreasing if $A_1\supseteq A_2\supseteq\cdots$

\end{definition}

\begin{definition}{Limit of Events}{Limit of Events}
Let $(\Omega,\ma, P)$ be a probability space.\\
(1) The limit $\limn A_n$ of an increasing sequence $\{A_1,A_2,\cdots\}$ of events in $A$ is the event that at least one of the events occurs, i.e., $$\limn A_n=\bigcup_{n=1}^\infty A_n.$$
(2) The limit $\limn A_n$ of a decreasing sequence $\{A_1,A_2,\cdots\}$ of events in $A$ is the event that all the events occur, i.e., $$\limn A_n=\bigcap_{n=1}^\infty A_n.$$

\end{definition}

\begin{theorem}{Continuity of probability measure}{Continuity}
Let $(\Omega,\ma, P)$ be a probability space.\\
(1) Suppose that $\{A_1,A_2,\cdots\}$ is an increasing sequence of events in $A$. Then $$P\left(\limn A_n\right)=\limn P(A_n).$$
(2) Suppose that $\{A_1,A_2,\cdots\}$ is a decreasing sequence of events in $A$. Then $$P\left(\limn A_n\right)=\limn P(A_n).$$
\end{theorem}

\begin{remark}
If $P(A)=0$, then it is not necessary that $A=\Phi$, e.g., $\Omega=(0,1)$ and $A=A_\alpha,\ \alpha\in(0,1)$.
If $P(A)=1$, then it is not necessary that $A=\Omega$, e.g., $\Omega=(0,1)$ and $A=A_\alpha^c,\ \alpha\in(0,1)$.

\end{remark}

\begin{definition}{Length}{Length}
The length of the intervals $(a,b),\ [a,b),\ (a,b],\ [a,b]$ are defined to be $(b-a)$.
\end{definition}

\begin{definition}{Random}{Random}
A point is said to be randomly selected from an interval $(a,b)$ if any subintervals of $(a,b)$ with the same length are equally likely to contain the randomly selected point.
\end{definition}

\begin{theorem}{Probability of Randomness}{Probability of Randomness}
The probability that a randomly selected point from $(a,b)$ falls in the subinterval $(\alpha,\beta)$ of $(a,b)$ is $$\frac{\beta-\alpha}{b-a}.$$

\end{theorem}

\begin{definition}{Borel Algebra}{Borel Algebra}
The $\sigma$-algebra of subsets of $(a,b)$ generated by the set of all subintervals of $(a,b)$ is called Borel algebra associated with $(a,b)$ and is denoted $\mathcal{B}_{(a,b)}$.
\end{definition}

\begin{theorem}{Existence of Probability Measure}{Existence}
For any interval $(a,b)$, there exists a unique probability measure $P$ on $\mathcal{B}_{(a,b)}$ s.t., $$P\left[(\alpha,\beta)\right]=\frac{\beta-\alpha}{b-a},$$
for all $(\alpha,\beta)\subseteq(a,b)$.
\end{theorem}



%%=====模板========================================
%%================================================

\chapter{Templates}
\begin{definition}{}{}

\end{definition}

\begin{theorem}{}{}

\end{theorem}

\begin{corollary}{}{}

\end{corollary}

\begin{lemma}{}{}

\end{lemma}

\begin{proof}

\end{proof}

\begin{remark}

\end{remark}


\end{document}
